\documentclass[14pt, a4paper]{extreport}
\usepackage[a4paper, total={6in, 9in}]{geometry}
\usepackage[utf8]{inputenc}
\usepackage[T2A]{fontenc}
\usepackage[russian]{babel}
\usepackage{fancyhdr}
\usepackage{amsmath}

\pagestyle{fancy}
\fancyhf{}
\renewcommand{\headrulewidth}{0pt}

\setcounter{page}{444}

\rfoot{
\begin{center}
\line(1, 0){100}\\
\textsl{\thepage}
\end{center}
}

\begin{document}
\par \noindent лите это понятие), лежащей в соприкасающейся плоскости с центром в центре кривизны кривой и радиусом, равным радиусу кривизны в точке $\mathscr r({t_0})$. \\
\par Это предельная окружность называется \textit{соприкасающейся окружностью} в данной точке кривой.\\
\par \noindent \large \textsf{\textbf{17.6. Эвольвента}}\\
\par \noindent \normalsize Как известно, $\frac{d\small \textit{\textbf{t}}}{ds} = \mathscr k\textit{\textbf{n}}$. Покажем, что для плоских кривых\\
\par
\par \hfill $\frac{d\textit{\textbf{n}}}{ds}$ = -$\mathscr k\textit{\textbf{t}}$. \hfill (17.31)\\
\par В самом деле, поскольку $\mathscr\textit{\textit{\textbf{n}}}$ - единичный вектор и, следовательно, имееь постоянную длину, его производная $\frac{d\textit{\textbf{n}}}{ds}$ перпендикулярна ему. Касательный вектор $\mathscr \textit{\textbf{t}}$ также перпендикулярен вектору $\mathscr {\textit{\textbf{n}}}$. На плоскости два вектора, перпендикулярные третьему, коллинеарны, поэтому\\

\par \hfill $\frac{d\textit{\textbf{n}}}{ds} = a\textit{\textbf{t}}$. \hfill (17.32)\\

\par \noindent Для того чтобы найти значение коэффициента $\mathscr a$, продифференцируем по длине дуги тождество $\mathscr \textit{\textbf{tn}} = 0$. В результате получим\\

\par \begin{center} $\frac{dt}{ds}$\textbf{n} + \textbf{t}$\mathscr \frac{dn}{ds}$ = 0. \end{center} \\

\par \noindent Подставив сюда значения $\mathscr \frac{dt}{ds} = k\textit{\textbf{n}}, \frac{dn}{ds} = a\textit{\textbf{t}}$ и заметив, что $\mathscr \textit{\textbf{tt}} = \textit{\textbf{nn}} = 1$, получим $\mathscr a = -k.$  Отсюда, в силу равенства (17.32), и следует формула (17.31). Формулы (17.9) и (17.31), т.е. \\

\par \begin{center} $\mathscr \frac{dt}{ds} = k\textit{\textbf{n}}, \frac{dn}{ds} = -k\textit{\textbf{t}},$ 
\end{center} \\

\par \noindent называются формулами Френe$^1$ для плоской кривой.\\

\par \noindent \textsf{\textbf{Определение 8.}} \emph{Если кривая} Г$_1$ \emph{является эволютной плоской кривой} Г\emph{, то кривая} Г \emph{называется эволбвентной кривой} Г$_1$.}\\

\footnotetext[1]{Ж.Ф.Френе (1816-1900) - французский математик.}

\end{document}

