\documentclass[12pt]{article}
\usepackage[utf8]{inputenc}
\usepackage[russian]{babel}

\begin{document}
\setcounter{page}{444}

\par \noindent лите это понятие), лежащей в соприкасающейся плоскости с центром в центре кривизны кривой и радиусом, равным радиусу кривизны в точке $\mathscr r({t_0})$. \\

\par Это предельная окружность называется \textit{соприкасающейся окружностью} в данной точке кривой.\\

\par \noindent \large \textsf{\textbf{17.6. Эвольвента}}\\

\par \noindent \normalsize Как известно, $\frac{d\small \textbf{t}}{ds} = \mathscr k\textbf{n}$. Покажем, что для плоских кривых\\

\par \hfill $\frac{d\textbf{n}}{ds}$ = -$\mathscr k\textbf{t}$. \hfill (17.31)\\

\par В самом деле, поскольку $\mathscr\textbf{n}$ - единичный вектор и, следовательно, имееь постоянную длину, его производная $\frac{d\textbf{n}}{ds}$ перпендикулярна ему. Касательный вектор $\mathscr \textbf{t}$ также перпендикулярен вектору $\mathscr {\textbf{n}}$. На плоскости два вектора, перпендикулярные третьему, коллинеарны, поэтому\\

\par \hfill $\frac{d\textbf{n}}{ds} = a\textbf{t}$. \hfill (17.32)\\

\par \noindent Для того чтобы найти значение коэффициента $\mathscr a$, продифференцируем по длине дуги тождество $\mathscr \textbf{tn} = 0$. В результате получим\\

\par \begin{center} $\frac{dt}{ds}$\textbf{n} + \textbf{t}$\mathscr \frac{dn}{ds}$ = 0. \end{center} \\

\par \noindent Подставив сюда значения $\mathscr \frac{dt}{ds} = k\textbf{n}, \frac{dn}{ds} = a\textbf{t}$ и заметив, что $\mathscr \textbf{tt} = \textbf{nn} = 1$, получим $\mathscr a = -k.$  Отсюда, в силу равенства (17.32), и следует формула (17.31).
\indent Формулы (17.9) и (17.31), т.е.
\\

\par \begin{center} $\mathscr \frac{dt}{ds} = k\textbf{n}, \frac{dn}{ds} = -k\textbf{t},$ \end{center} \\

\par \noindent называются формулами Френe$^1$ для плоской кривой.\\

\par \noindent \textsf{\textbf{Определение 8.}} \emph{Если кривая} Г$_1$ \emph{является эволютной плоской кривой} Г\emph{, то кривая} Г \emph{называется эволбвентной кривой} Г$_1$.}\\

\footnotetext[1]{Ж.Ф.Френе (1816-1900) - французский математик.}

\end{document}
